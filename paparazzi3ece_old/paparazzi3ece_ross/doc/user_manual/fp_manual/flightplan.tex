%
%
%
%   $Id: flightplan.tex,v 1.1 2006/01/29 21:14:32 disciple Exp $
%   Copyright (C) 2003 Pascal Brisset, Antoine Drouin
%
% This file is part of paparazzi.
%
% paparazzi is free software; you can redistribute it and/or modify
% it under the terms of the GNU General Public License as published by
% the Free Software Foundation; either version 2, or (at your option)
% any later version.
%
% paparazzi is distributed in the hope that it will be useful,
% but WITHOUT ANY WARRANTY; without even the implied warranty of
% MERCHANTABILITY or FITNESS FOR A PARTICULAR PURPOSE.  See the
% GNU General Public License for more details.
%
% You should have received a copy of the GNU General Public License
% along with paparazzi; see the file COPYING.  If not, write to
% the Free Software Foundation, 59 Temple Place - Suite 330,
% Boston, MA 02111-1307, USA.  
% 



\documentclass{article}

\usepackage{a4wide}
\usepackage{graphicx}
\usepackage{makeidx}
\usepackage[pagebackref=true,hyperindex=true]{hyperref}
\usepackage{color}

\title{Paparazzi flight plan language manual}
\author{Louis Dugrain}
\date{\today}
\setlength{\parindent}{0pt}

\makeindex
%\newenvironment{example}{\linespread{0.8}}{\linespread{1}}
\renewcommand{\tt}[1]{\texttt{#1}}
\newcommand{\ex}[1]{\colorbox[gray]{0.92}{\tt{#1}}}
\newcommand{\hs}[1]{\hspace*{#1cm}}
\newcommand{\until}{%
	\tt{until} is written like the \tt{\hyperlink{exceptions}{exceptions}}
		\tt{\hyperlink{cond}{cond}}. It works like an
		\tt{\hyperlink{exceptions}{exception}} but only for \tt{stages}.
		When it's true the autopilot apply following \tt{stage}.
}
\definecolor{gris75}{gray}{0.25}
\newcommand{\qt}[1]{\textcolor{gris75}{#1}}

\begin{document}

\maketitle


\vspace{\stretch{1}}
%===============================================================================
\hypertarget{flightplanstructure}{\section{Structure of flight plan files}}
%===============================================================================

\begin{minipage}[ctb]{\textwidth}
\tt{\hyperlink{generaldata}{flight\_plan} \emph{name lat0 lon0 ground\_alt security\_height QFU alt max\_dist\_from\_home}} \\
	\hs{0.5} \hyperlink{rccontrol}{\tt{rc\_control}} \\
		\hs{1} \tt{mode AUTO1} \\
			\hs{1.5} [\tt{setting \emph{var range rc type}}]+ \\
		\hs{1} \tt{mode AUTO2} \\
			\hs{1.5} [\tt{setting \emph{var range rc type}}]+ \\
			
	\hs{0.5} [\hyperlink{dlsettings}{\tt{dl\_settings}}]? \\
		\hs{1} [\tt{dl\_setting \emph{var min max [step]}}]* \\
		
	\hs{0.5} \hyperlink{waypoints}{\tt{waypoints}} \\
		\hs{1} [\tt{waypoint \emph{name x y [lat] [lon] [alt]}}]+ \\
		
	\hs{0.5} [\hyperlink{include}{\tt{include}} \tt{\emph{name \hyperlink{procedures}{procedure} [x] [y] [rotate]}} \\
		\hs{1} [\tt{arg \emph{name value}}]* \\
		\hs{1} [\tt{with \emph{from to}}]* \\
	\hs{0.5} ]* \\
	
	\hs{0.5} \hyperlink{blocks}{\tt{blocks}} \\
		\hs{1} [\tt{block \emph{name [description]}}]+ \\
			\hs{1.5} \hyperlink{exceptions}{\tt{exception}} \tt{\emph{\tt{\hyperlink{cond}{cond}} deroute}} \\
			\hs{1.5} \hyperlink{heading}{\tt{heading}} \tt{\emph{course [\hyperlink{verticalmodes}{vmode}] [alt] [gaz] [climb] [pitch] until}} \\
			\hs{1.5} \hyperlink{attitude}{\tt{attitude}} \tt{\emph{roll [\hyperlink{verticalmodes}{vmode}] [alt] [gaz] [climb] [pitch] until}} \\
			\hs{1.5} \parbox[]{13cm}{\tt{\hyperlink{go}{go} \emph{wp [wp\_qdr wp\_dist] [from] [from\_qdr from\_dist] [\hyperlink{horizontalmodes}{hmode}] [\hyperlink{verticalmodes}{vmode}] [alt] [pitch] [approaching\_time] [gaz] [\hyperlink{cam}{cam\_mode}] [\hyperlink{cam}{target}] [\hyperlink{cam}{cam\_ac\_target}] [until]}}} \\
			\hs{1.5} \hyperlink{follow}{\tt{follow}} \tt{\emph{ac\_id distance height [\hyperlink{cam}{cam\_mode}] [\hyperlink{cam}{target}] [\hyperlink{cam}{cam\_ac\_target}]}} \\
			\hs{1.5} \parbox[]{13cm}{\tt{\hyperlink{circle}{circle} \emph{wp [wp\_qdr wp\_dist] radius [alt] [\hyperlink{verticalmodes}{vmode}] [\hyperlink{cam}{cam\_mode}] [\hyperlink{cam}{target}] [\hyperlink{cam}{cam\_ac\_target}] [climb] [until]}}} \\
			\hs{1.5} \hyperlink{stay}{\tt{stay}} \tt{\emph{wp [\hyperlink{verticalmodes}{vmode}] [gaz]}} \\
			\hs{1.5} \hyperlink{deroute}{\tt{deroute}} \tt{\emph{block}} \\
			\hs{1.5} \hyperlink{set}{\tt{set}} \tt{\emph{var value [until]}} \\
			\hs{1.5} \hyperlink{xyz}{\tt{xyz}} \tt{\emph{[radius] [\hyperlink{cam}{cam\_mode}] [\hyperlink{cam}{target}] [\hyperlink{cam}{cam\_ac\_target}]}} \\
			\hs{1.5} \hyperlink{while}{\tt{while}} \tt{\emph{[\tt{\hyperlink{cond}{cond}}]}} \\
				\hs{2} [\tt{while}, \tt{heading}, \tt{attitude}, \tt{go}, \tt{xyz}, \tt{set}, \tt{circle}, \tt{deroute}, \tt{stay}]* \\
\end{minipage}

\vspace{\stretch{1}}
\pagebreak


%===============================================================================
\section{Description}
%===============================================================================

A flight plan is devided in 6 parts:
\begin{itemize}
	\item some general data about the mission
	\item \hyperlink{rccontrol}{\textbf{\tt{rc\_control}}:} radio control functions' description
	\item \hyperlink{dlsettings}{\textbf{\tt{dl\_settings}:}} \emph{\textbf{I don't know and must complete this}}
	\item \hyperlink{waypoints}{\textbf{\tt{waypoints}:}} a list of waypoint used for navigation
	\item \hyperlink{include}{\textbf{\tt{include}:}} a \hyperlink{procedures}{procedure} with its arguments
	\item \hyperlink{blocks}{\textbf{\tt{blocks}:}} the description of the mission
\end{itemize}


%===============================================================================
\hypertarget{generaldata}{\subsection{General data}}
%===============================================================================

\begin{quote}
	\tt{<flight\_plan \emph{name lat0 lon0 ground\_alt security\_height QFU alt max\_dist\_from\_home}>}
\end{quote}
\par
\par

In the arguments of first meta tag \tt{flight\_plan} you can find several general data about mission:
\begin{itemize}
	\item firstly there is the \tt{name} of the mission,
	\item then \tt{lat0} and \tt{lon0} describe the latitude and longitude of the point \{0,0\} in WGS84.
	\item \tt{ground\_alt} is the ground altitude%
	\footnote{which is in meter set in relation whith sea level} (in meter).
	It's used to easily place waypoints altitude in relation with \tt{GROUND\_ALT},
	\item \tt{security\_height} is the height below which the UAV \textbf{stop is engine},
	\item the \tt{QFU} is a kind of global constant often used during takeoff ot landing. It is the magnetic heading of the runway, generaly the opposite of wind direction
	\item \tt{alt} is the default altitude of waypoints when it's not specify (see \autoref{waypoints2})
	\item \tt{max\_dist\_from\_home} describe a circle (in meter) outside which the UAV come back to \tt{HOME} waypoint.
\end{itemize}

\begin{minipage}[ctb]{\textwidth}
Here is an example of the first line of a flight plan:
\begin{quote}
	\ex{<flight\_plan NAME="\qt{example - Muret1}"} \\
	\ex{\hs{2.2} LAT0="\qt{43.46223}" LON0="\qt{1.27289}"} \\
	\ex{\hs{2.2} MAX\_DIST\_FROM\_HOME="\qt{300}"} \\
	\ex{\hs{2.2} GROUND\_ALT="\qt{185}"} \\
	\ex{\hs{2.2} SECURITY\_HEIGHT="\qt{25}"} \\
	\ex{\hs{2.2} QFU="\qt{270}" ALT="\qt{250}">} \\
\end{quote}
\end{minipage}


%===============================================================================
\hypertarget{rccontrol}{\subsection{\tt{rc\_control}}}
%===============================================================================
Then, there is the radio control section which describes what variables can be changed during flight and by which way you can change it.
This is very usefull when you want to adjust trim or loop's gain 

You must specify for both \tt{mode}, \tt{AUTO1} and \tt{AUTO2}, the settings by this way:
	\begin{quote}
		\tt{<setting \emph{var range rc type}/>}
	\end{quote}
\par

where:
\begin{itemize}
	\item \tt{var} is the variable's name you want to change during flight
	\item \tt{type} is its type
	\item \tt{range} is the max range you can change by one full change on the radio control
	\item \tt{rc} specify the radio control voice you'll use to change this variable \\
\end{itemize}

\begin{minipage}[ctb]{\textwidth}
Here is an example of the \hyperlink{rccontrol}{\tt{rc\_control}} section:
\begin{quote}
	\ex{<rc\_control>} \\
	\ex{\hs{0.5} <mode NAME="\qt{AUTO1}">} \\
	\ex{\hs{1} <setting VAR="\qt{ir\_pitch\_neutral}" RANGE="\qt{60.}"} \\
	\ex{\hs{2.7} RC="\qt{gain\_1\_up}" TYPE="\qt{int16}"/>} \\
	\ex{\hs{1} <setting VAR="\qt{ir\_roll\_neutral}" RANGE="\qt{-60.}"} \\
	\ex{\hs{2.7} RC="\qt{gain\_1\_down}" TYPE="\qt{int16}"/>} \\
	\ex{\hs{0.5} </mode>} \\
	\ex{\hs{0.5} <mode NAME="\qt{AUTO2}">} \\
	\ex{\hs{1} <setting VAR="\qt{course\_pgain}" RANGE="\qt{0.1}"} \\
	\ex{\hs{2.7} RC="\qt{gain\_1\_up}" TYPE="\qt{float}"/>} \\
	\ex{\hs{1} <setting VAR="\qt{pitch\_of\_roll}" RANGE="\qt{.2}"} \\
	\ex{\hs{2.7} RC="\qt{gain\_1\_down}" TYPE="\qt{float}"/>} \\
	\ex{\hs{0.5} </mode>} \\
	\ex{</rc\_control>} \\
\end{quote}
\end{minipage}


%===============================================================================
\hypertarget{dlsettings}{\subsection{\tt{dl\_settings}}}
%===============================================================================
I don't know what it is. Maybe I should...  :-)


%===============================================================================
\hypertarget{waypoints}{\subsection{\tt{waypoints}}}
%===============================================================================
\label{waypoints2}
You describe here the waypoint you'll use in your flight plan. \par
The syntax is following:
	\begin{quote}
		\tt{<waypoint \emph{name x y [lat] [lon] [alt]}/>}
	\end{quote}
\par
where \tt{x} and \tt{y} are the coordinates in meter from point \{0,0\}%
\footnote{in an UTM projection}. \tt{lat} and \tt{lon} are now useless because the transformation in WGS84 coordinates is automatically done. \tt{alt} can be specify. If not, the default altitude is taken.
\\


There must always be a waypoint \tt{HOME} where the UAV can come back if there is any problem.
\\


\begin{minipage}[ctb]{\textwidth}
Here is the result:
\begin{quote}
	\ex{<waypoints>} \\
	\ex{\hs{0.5} <waypoint name="\qt{HOME}" x="\qt{0.0}" y="\qt{30.0}"/>} \\
	\ex{\hs{0.5} <waypoint name="\qt{1}" x="\qt{-100.0}" y="\qt{60.0}"
	alt="\qt{250.}"/>} \\
	\ex{\hs{0.5} <waypoint name="\qt{2}" x="\qt{-130.0}" y="\qt{217.5}"
	alt="\qt{250.}"/>} \\
	\ex{\hs{0.5} <waypoint name="\qt{3}" x="\qt{25.0}" y="\qt{285.0}"
	alt="\qt{250.}"/>} \\
	\ex{\hs{0.5} <waypoint name="\qt{4}" x="\qt{162.5}" y="\qt{210.0}"
	alt="\qt{250.}"/>} \\
	\ex{\hs{0.5} <waypoint name="\qt{5}" x="\qt{97.5}" y="\qt{60.0}"
	alt="\qt{250.}"/>} \\
	\ex{</waypoints>}
\end{quote}
\end{minipage}
	

%===============================================================================
\hypertarget{include}{\subsection{\tt{include}}}
%===============================================================================
\label{include2}
\tt{include} is used to add all \hyperlink{blocks}{\tt{blocks}} defined in a
\hyperlink{procedures}{\tt{procedure}}. It's usefull to use ever made
\hyperlink{procedures}{\tt{procedure}} with only few arguments and then
clarify the flight plan. \\


Here is the structure:
\begin{quote}

	\tt{<include \emph{name \hyperlink{procedures}{procedure} [x] [y] [rotate]}>} \\
	\hs{0.5} [\tt{<arg \emph{name value}/>}]* \\
	\hs{0.5} [\tt{<with \emph{from to}/>}]* \\
	\tt{</include>}
\end{quote}

\tt{name} will be used in this flight plan to call the
\hyperlink{procedures}{\tt{procedure}}, which known by its relative path. The
\hyperlink{procedures}{\tt{procedure}} can be moved of \tt{x} and \tt{y}
and rotate with \tt{rotate} argument.\\
We pass the argument \tt{name} and their \tt{value} in the \tt{arg} lines and
the events \tt{from} and the block to switch to \tt{to} in the \tt{with}
line. \\


Then, each \hyperlink{blocks}{\tt{blocks}} of the
\hyperlink{procedures}{\tt{procedure}}
is like any \hyperlink{blocks}{\tt{block}} of the flight plan and can be called
with \tt{deroute} command: \ex{deroute="\qt{hippo1.one\_block}"}, where
\tt{one\_block} is one of the \hyperlink{blocks}{\tt{block}} of the
\hyperlink{procedures}{\tt{procedure}} locally called \tt{hippo1}. \\
There are more explainations in \autoref{procedureuse2}. \\


\begin{minipage}[ctb]{\textwidth}
Here is an example:
\begin{quote}
	\ex{<include name="\qt{hippo1}" procedure="\qt{hippo.xml}" x="\qt{-100}" y="\qt{150}" rotate="\qt{0}">} \\
	\ex{\hs{0.5} <arg name="\qt{alt}" value="\qt{GROUND\_ALT+100}"/>} \\
	\ex{\hs{0.5} <with from="\qt{event1}" to="\qt{penta}"/>} \\
	\ex{</include>}
\end{quote}
\end{minipage}


%===============================================================================
\hypertarget{blocks}{\subsection{\tt{blocks}}}
%===============================================================================
\tt{Blocks} are the main part of a flight plan because they describe each 
main stages of the mission. They are made of various primitives, called
\tt{stages}, you can put one after the other. When a \tt{stage} is finished,
the autopilot go to next one.


%-------------------------------------------------------------------------------
\hypertarget{exceptions}{\subsubsection{\tt{exceptions}}}
%-------------------------------------------------------------------------------
In order to switch between
\tt{blocks}, there are \tt{exceptions} which deroute the mission
on another \tt{block} by the \tt{deroute} argument when the
\tt{cond}ition is true (\tt{deroute} must be the name of
a \tt{block}). Here is the syntax of \tt{exceptions}:

\begin{quote}
	\tt{<exception \emph{cond deroute}/>}
\end{quote}

\tt{\hypertarget{cond}{cond}} are annalysed whith some functions:

\begin{itemize}
	\item \tt{<}, \tt{>}, \tt{<=}, \tt{>=}, \tt{<>}, \tt{==}, \tt{+}, \tt{-}, \tt{/}, \tt{*}
	\item \tt{And(\emph{cond1}, \emph{cond2})} and \tt{Or(\emph{cond1}, \emph{cond2})}
	\item \tt{RcEvent1()} and \tt{RcEvent2()}, which correspond
	to radio control event
	\item \tt{Qdr(\emph{value})}, which is true when Qdr is more or less
	of 10 degrees of the \emph{value}.
\end{itemize}

There are also variables:
\begin{itemize}
	\item \tt{launch} is true when the UAV think it has been launched
	\item \tt{estimator\_z} is altitude
	\item \tt{estimator\_flight\_time}
	\item \tt{estimator\_hspeed\_mod} is the horizontal speed (in m/s)
	\item \tt{stage\_time} is the time past since the begining of the stage
	\item \tt{block\_time} is the time past since the entrance in the
	last block
	\item \tt{GROUND\_ALT}, \tt{QFU} and \tt{SECURITY\_HEIGHT}
	\item \tt{TRUE} to make infinite loop \\
\end{itemize}


Here there are some example of \tt{exceptions}:
\begin{quote}
	\ex{<exception cond="\qt{(RcEvent1())}" deroute="\qt{go\_again}"/>} \\
	\ex{<exception cond="\qt{(GROUND\_ALT+10 > estimator\_z)}" deroute="\qt{go\_again}"/>} \\
	\ex{<exception cond="\qt{(RcEvent2())}" deroute="\qt{stop}"/>} \\
	\ex{<exception cond="\qt{(estimator\_flight\_time > 840)}" deroute="\qt{quick\_land}"/>}
\end{quote}

Now, we are studying the other primitive. Firstly, we'll see the loop.


%-------------------------------------------------------------------------------
\hypertarget{while}{\subsubsection{\tt{while}}}
%-------------------------------------------------------------------------------
The syntax is following:
\begin{quote}
	\tt{while \emph{[\hyperlink{cond}{cond}]}}
\end{quote}
Almost any other primitive can be put in a short or long loop. \\
\tt{\hyperlink{cond}{cond}} is written like the \tt{\hyperlink{exceptions}{exceptions}} \tt{\hyperlink{cond}{cond}}.
\\


\begin{minipage}[ctb]{\textwidth}
Here is an example of an infinite hippodrome between \tt{c1} and \tt{c2}:
\begin{quote}
	\ex{<while>} \\
	\ex{\hs{0.5} <go hmode="\qt{route}" alt="\qt{alt}"} \\ 
	\ex{\hs{1.2} from="\qt{c1}" from\_qdr="\qt{0}" from\_dist="\qt{radius}"} \\
	\ex{\hs{1.2} wp="\qt{c2}" wp\_qdr="\qt{0}" wp\_dist="\qt{radius}"/>} \\
	\ex{\hs{0.5} <circle wp="\qt{c2}" radius="\qt{radius}" until="\qt{Qdr(180)}" alt="\qt{alt}"/>} \\
	\ex{\hs{0.5} <go hmode="\qt{route}" alt="\qt{alt}"} \\
	\ex{\hs{1.2} from="\qt{c2}" from\_qdr="\qt{180}" from\_dist="\qt{radius}"} \\
	\ex{\hs{1.2} wp="\qt{c1}" wp\_qdr="\qt{180}" wp\_dist="\qt{radius}"/>} \\
	\ex{\hs{0.5} <circle wp="\qt{c1}" radius="\qt{radius}" until="\qt{Qdr(0)}" alt="\qt{alt}"/>} \\
	\ex{</while>}
\end{quote}
\end{minipage}


%-------------------------------------------------------------------------------
\hypertarget{modes}{\subsubsection{Various mode}}
%-------------------------------------------------------------------------------
\label{modes2}
There are various modes, separated into two groups (horizontal and vertical).
Each group is divide up into 3 modes, corresponding to the three level loop
in the autopilot.

Here is the various modes:
\begin{itemize}
	\item \hypertarget{horizontalmodes}{horizontal modes (\tt{hmode})}:
	\begin{itemize}
		\item You just specify \tt{roll}:
		\ex{roll="\qt{0.2}"}
		\item You ask an \tt{course} with the
		\hyperlink{heading}{\tt{heading}} primitive:\\
		\ex{<heading course="\qt{QFU}" until="\qt{(estimator\_z > SECURITY\_ALT)}"/>}
		\item You want the UAV navigate. There are 5 primitives we'll study further:
		\begin{itemize}
			\item \tt{\hyperlink{go}{go}}
			\item \tt{\hyperlink{circle}{circle}}
			\item \tt{\hyperlink{follow}{follow}}
			\item \tt{\hyperlink{stay}{stay}}
			\item \tt{\hyperlink{xyz}{xyz}}
		\end{itemize}
	\end{itemize}
	\item \hypertarget{verticalmodes}{vertical modes (\tt{vmode})}:
	\begin{itemize}
		\item You ask a \tt{pitch}: \ex{pitch="\qt{0.1}"}
		and \tt{gaz}: \ex{vmode="\qt{gaz}" gaz="\qt{0.5}"}
		(\tt{gaz} are in percent).
		\item You ask a \tt{climb}: \ex{vmode="\qt{climb}" climb="\qt{1}"}
		where \tt{climb} is in meter per second.
		
		\item You want the UAV keep an altitude with the \tt{alt} argument:\\
		\ex{vmode="\qt{alt}" alt="\qt{GROUND\_ALT+50}"}
		\item You ask a \tt{glide} which is a special mode (a derivative of
		\tt{alt} mode) where you need also be in \tt{route hmode}:
		\ex{hmode="\qt{route}" vmode="\qt{glide}"}.
	\end{itemize}
\end{itemize}

In vertical modes, there are two ways to specify what you want:
\begin{itemize}
	\item You set \tt{pitch} (or don't mention it which means it will be
	set to zero) and let the \tt{gaz} automaticly set to respect the
	\tt{climb} or \tt{alt} command. It's the default vertical mode.
	You just have to set the \tt{pitch} if you don't want it set to
	zero: \ex{\tt{pitch="\qt{0.1}"}}.
	
	\item You set \tt{gaz} and let the \tt{pitch} automaticly set to
	respect the \tt{climb} or \tt{alt} command. This is the auto
	\tt{pitch} mode: \ex{\tt{pitch="\qt{auto}"}}.
	\tt{gaz} must be set:	\ex{\tt{gaz="\qt{0.5}"}}.
\end{itemize}


%-------------------------------------------------------------------------------
\hypertarget{attitude}{\subsubsection{\tt{attitude}}}
%-------------------------------------------------------------------------------
\tt{attitude} is the command which corresponds to the first level loop for
\hyperlink{horizontalmodes}{horizontal mode} in the autopilot. \\
Its syntax is following:

\begin{quote}
	\tt{attitude \qt{roll [\hyperlink{verticalmodes}{vmode}] [alt] [gaz] [climb] [pitch] until}}
\end{quote}

Where \until \\

Here is an example:
\begin{quote}
	\ex{<attitude roll="\qt{0}" pitch="\qt{0}" vmode="\qt{gaz}" gaz="\qt{0.5}" until="\qt{(FALSE)}"/>}
\end{quote}


%-------------------------------------------------------------------------------
\hypertarget{heading}{\subsubsection{\tt{heading}}}
%-------------------------------------------------------------------------------
\tt{heading} primitive corresponds to the second level loop for
\hyperlink{horizontalmodes}{horizontal mode} in the autopilot. \\
Its syntax is following:

\begin{quote}
	\tt{heading \emph{course [\hyperlink{verticalmodes}{vmode}] [alt] [gaz] [climb] [pitch] until}}
\end{quote}

Where \until \\

Here is an example:
\begin{quote}
	\ex{<heading vmode="\qt{gaz}" gaz="\qt{0.8}" pitch="\qt{0.15}" course="\qt{QFU}"} \\
	\ex{\hs{1.5} until="\qt{(estimator\_z > SECURITY\_ALT)}"/>}
\end{quote}


%-------------------------------------------------------------------------------
\hypertarget{go}{\subsubsection{\tt{go}}}
%-------------------------------------------------------------------------------
\tt{go} primitive corresponds to the third level loop for
\hyperlink{horizontalmodes}{horizontal mode} in the autopilot. \\
Its syntax is following:

\begin{quote}
	\tt{go \emph{wp [wp\_qdr wp\_dist] [from] [from\_qdr from\_dist]
	[\hyperlink{horizontalmodes}{hmode}] [\hyperlink{verticalmodes}{vmode}]
	[alt] [pitch] [approaching\_time] [gaz] [\hyperlink{cam}{cam\_mode}]
	[\hyperlink{cam}{target}] [\hyperlink{cam}{cam\_ac\_target}] [until]}}
\end{quote}


where:
\begin{itemize}
	\item \tt{wp} and \tt{from} are \hyperlink{waypoints}{waypoint} valid names.
	\tt{wp} is the \hyperlink{waypoints}{waypoint} the UAV must go to.
	\tt{from} is the \hyperlink{waypoints}{waypoint} the UAV come from if
	you ask a \tt{route} \hyperlink{horizontalmodes}{horizontal mode}\\
	(\ex{\tt{hmode="\qt{route}"}}).
	In \tt{route} mode, the UAV will follow a path between the two waypoints
	\tt{from} and \tt{wp}. If it is not specify, the UAV will go directly
	to \tt{wp} (probably by doing a dog courb).
	
	\item \tt{wp\_qdr} and \tt{wp\_dist} are used to create a ghost waypoint
	defined from \tt{wp} by its QDR and distance (\ex{from="\qt{wp1}"
	from\_qdr="\qt{0}" from\_dist="\qt{60}"} is a 60 meters northward new
	waypoint).
	
	\item \tt{from\_qdr} and \tt{from\_dist} are used by the same way to create
	a ghost waypoint defined from \tt{from}.
	
	\item \hyperlink{verticalmodes}{\tt{vmode}}, \tt{alt} and \tt{pitch} are
	used like it's explained in \autoref{modes2}.
	
	\item \tt{approaching\_time} is a way to specify when the UAV decide
	is near enough of the \tt{wp} to do the next \tt{stage}. By default,
	\tt{approaching\_time} is egal to \tt{CAROTT} time, which is by
	default 5 seconds.
	
	\item \tt{cam\_mode}, \tt{target} and \tt{cam\_ac\_target} are used as
	specified in \autoref{cam2}.
	
	\item \until	
\end{itemize}

\emph{Note:} it is not obliged to specify the waypoint \tt{from} in \tt{route}
mode if you have ever used a \tt{go} primitive just the previous \tt{stage}.
For example, with this command:
\begin{quote}
	\ex{<go wp="\qt{wp2}" from="\qt{wp1}" hmode="\qt{route}"/>} \\
	\ex{<go wp="\qt{wp3}" hmode="\qt{route}"/>}
\end{quote}
the autopilot will understand: follow the route between \tt{wp1} and \tt{wp2}.
When you are near \tt{wp2}, follow the route between \tt{wp2} and \tt{wp3}.
In each branch, the altitude is the altitude of the going to waypoint.
\\


\begin{minipage}[ctb]{\textwidth}
Here is an example of primitive \tt{go}'s use:
\begin{quote}
	\ex{<go hmode="\qt{route}" alt="\qt{GROUND\_ALT+50}"} \\ 
	\ex{\hs{0.58} from="\qt{wp\_left}" from\_qdr="\qt{0}" from\_dist="\qt{60}"} \\
	\ex{\hs{0.58} wp="\qt{wp\_right}" wp\_qdr="\qt{0}" wp\_dist="\qt{60}"} \\
	\ex{\hs{0.58} pitch="\qt{auto}" gaz="\qt{0.5}"} \\
	\ex{\hs{0.58} cam\_mode="\qt{target}" target="\qt{my\_house}"/>} \\
\end{quote}
\end{minipage}


%-------------------------------------------------------------------------------
\hypertarget{circle}{\subsubsection{\tt{circle}}}
%-------------------------------------------------------------------------------
Like \hyperlink{go}{\tt{go}} primitive, \tt{circle} works with autopilot's
third level loop for \hyperlink{horizontalmodes}{horizontal mode}. \\
The syntax is following:

\begin{quote}
	\tt{circle \emph{wp [wp\_qdr wp\_dist] radius [alt]
	[\hyperlink{verticalmodes}{vmode}] [\hyperlink{cam}{cam\_mode}]
	[\hyperlink{cam}{target}] [\hyperlink{cam}{cam\_ac\_target}]
	[climb] [until]}}
\end{quote}


where:
\begin{itemize}
	\item \tt{wp} is one of the defined \hyperlink{waypoints}{waypoint}.
	The UAV will turn around with a \tt{radius} radius circle.
	
	\item \tt{wp\_qdr} and \tt{wp\_dist} are used to create a ghost waypoint
	defined from \tt{wp} by its QDR and distance (\ex{from="\qt{wp1}"
	from\_qdr="\qt{0}" from\_dist="\qt{60}"} is a 60 meters northward
	new waypoint).

	\item \hyperlink{verticalmodes}{\tt{vmode}}, \tt{alt} and \tt{climb} are
	used like it's explained in \autoref{modes2}.

	\item \tt{cam\_mode}, \tt{target} and \tt{cam\_ac\_target} are used as
	specified in \autoref{cam2}.

	\item \until \\
\end{itemize}


\begin{minipage}[ctb]{\textwidth}
Here is an example of primitive \tt{circle}'s use:
\begin{quote}
	\ex{<circle wp="\qt{HOME}" wp\_qdr="\qt{70}" wp\_dist="\qt{100}" radius="\qt{50}"} \\
	\ex{\hs{1.3} vmode="\qt{climb}" climb="\qt{-1}" cam\_mode="\qt{nadir}"} \\
	\ex{\hs{1.3} until="\qt{And(GROUND\_ALT+30 > estimator\_z, Qdr(180))}"/>} \\
\end{quote}
\end{minipage}


%-------------------------------------------------------------------------------
\hypertarget{follow}{\subsubsection{\tt{follow}}}
%-------------------------------------------------------------------------------
\begin{quote}
	\tt{follow \emph{ac\_id distance height [\hyperlink{cam}{cam\_mode}]
	[\hyperlink{cam}{target}] [\hyperlink{cam}{cam\_ac\_target}]}}
\end{quote}

\tt{follow} is a special primitive which make your UAV \tt{follow} the
\tt{ac\_id} UAV at \tt{distance} meters behind and \tt{height} meters above
(and always above \tt{SECURITY\_ALT}). \\

\tt{cam\_mode}, \tt{target} and \tt{cam\_ac\_target} can be used as
specified in \autoref{cam2}. \\

\begin{minipage}[ctb]{\textwidth}
Here is an example of \tt{follow} mode use:
\begin{quote}
	\ex{<follow ac\_id="\qt{4}" distance="\qt{50}" height="\qt{20}" cam\_mode="\qt{follow}" cam\_ac\_target="\qt{4}"/>} \\
\end{quote}
\end{minipage}


%-------------------------------------------------------------------------------
\hypertarget{stay}{\subsubsection{\tt{stay}}}
%-------------------------------------------------------------------------------
\begin{quote}
	\tt{stay \emph{wp [\hyperlink{verticalmodes}{vmode}] [gaz]}}
\end{quote}

\tt{stay} is a mode for helicopter-like UAVs to make them remain at a
valid \hyperlink{waypoints}{waypoint} \tt{wp}. \\

\begin{minipage}[ctb]{\textwidth}
Here is an example of \tt{stay} mode use:
\begin{quote}
	\ex{<stay wp="\qt{my\_house}"/>} \\
\end{quote}
\end{minipage}


%-------------------------------------------------------------------------------
\hypertarget{xyz}{\subsubsection{\tt{xyz}}}
%-------------------------------------------------------------------------------
\begin{quote}
	\tt{xyz \emph{[radius] [\hyperlink{cam}{cam\_mode}] [\hyperlink{cam}{target}]
	[\hyperlink{cam}{cam\_ac\_target}]}}
\end{quote}

\tt{xyz} is a special mode where the UAV circles around a user moveable
waypoint, called \tt{carrot} with a \tt{radius} radius.
By default, the \tt{radius} is 60 meters.
It is usefull to put this command in a special \hyperlink{blocks}{\tt{block}}
so as to let user to fly around a special target not determined at fligh
plan writing time. The user uses radio control to move the target waypoint. \\

\tt{cam\_mode}, \tt{target} and \tt{cam\_ac\_target} can be used as
specified in \autoref{cam2}. \\


\begin{minipage}[ctb]{\textwidth}
Here is an example of \tt{xyz} mode use:
\begin{quote}
	\ex{<block name="\qt{xyz}">} \\
	\ex{\hs{0.5} <exception cond="\qt{(RcEvent2())}" deroute="\qt{circlehome}"/>} \\
	\ex{\hs{0.5} <xyz radius="\qt{40}" cam\_mode="\qt{target}"/>} \\
	\ex{</block>}
\end{quote}
\end{minipage}


%-------------------------------------------------------------------------------
\hypertarget{cam}{\subsubsection{Camera options}}
%-------------------------------------------------------------------------------
\label{cam2}
There are 6 modes whith camera (\tt{cam\_mode}):
\begin{itemize}
	\item \tt{null}: there is no camera.
	\item \tt{fix}: camera is fix.
	\item \tt{manual}: camera is manually moved.
	\item \tt{nadir}: camera looks always just under the UAV, even if it is turning.
	\item \tt{target}: camera is set on \tt{target}. In \tt{xyz} mode, the
	camera is set on the \tt{carrot}, meaning the center of the circle.
	If \tt{target} is a valid \hyperlink{waypoints}{waypoint}. It keep view on it.
	Else it is a manually set target (meaning it moves automaticaly to keep
	view on the user defined target).
	\item \tt{follow}: camera is set on \tt{cam\_ac\_target} aircraft.
\end{itemize}


%-------------------------------------------------------------------------------
\hypertarget{set}{\subsubsection{\tt{set}}}
%-------------------------------------------------------------------------------
\tt{set} is not a primitive like the other ones. This command is used to
changed a variable with this syntax:

\begin{quote}
	\tt{set \emph{var value [until]}}
\end{quote}


It's usefull for developpement to set \tt{var} variable to \tt{value} to test
something new and \tt{set} it back to current \tt{value} in another
\hyperlink{blocks}{\tt{block}} you can allways switch to in order to prevent
crash. \\

\until \\

\begin{minipage}[ctb]{\textwidth}
Here is a little example of \tt{set} use where the developper wants to try a
new navigation algorithm by doing a route between \tt{wp1} and \tt{wp2}.
At any time, he can stop the new algorithm by switching to
\tt{stop\_test block} with a radio control \tt{RcEvent1} event and then go
in the \tt{come\_back block}.
\begin{quote}
	\ex{<block name="\qt{test}"} \\ 
	\ex{\hs{0.5} <exception cond="\qt{(RcEvent1())}" deroute="\qt{stop\_test}"/>} \\
	\ex{\hs{0.5} <set var="\qt{new\_navigation\_algorithm}" value="\qt{TRUE}"/>} \\
	\ex{\hs{0.5} <go wp="\qt{wp2}" from="\qt{wp1}" hmode="\qt{route}"/>} \\
	\ex{\hs{0.5} <deroute block="\qt{stop\_test}"/>} \\
	\ex{</block}\\
	\ex{<block name="\qt{stop\_test}"} \\ 
	\ex{\hs{0.5} <set var="\qt{new\_navigation\_algorithm}" value="\qt{FALSE}"/>} \\
	\ex{\hs{0.5} <deroute block="\qt{come\_back}"/>} \\
	\ex{</block}\\
\end{quote}
\end{minipage}


%-------------------------------------------------------------------------------
\hypertarget{deroute}{\subsubsection{\tt{deroute}}}
%-------------------------------------------------------------------------------
\tt{deroute} is not a primitive. It's just an easy way to specify which is the
following \hyperlink{blocks}{\tt{block}} in flight plan execution:
\begin{quote}
	\ex{<deroute block="\qt{another\_block}"/>}
\end{quote}


If there is no \tt{deroute} command at the end of a block, then the autopilot
will jump to the next one.


%===============================================================================
\hypertarget{procedures}{\section{\tt{procedures}}}
%===============================================================================
\tt{procedures} are used as fight plan library. You write it once and use it
when you want. It prevents you to think every time about a common figure like
an hippodrome.

%-------------------------------------------------------------------------------
\hypertarget{procedurestructure}{\subsection{Structure of \tt{procedures}}}
%-------------------------------------------------------------------------------
The way to write it is quite similar to flight plan. Here is the structure: \\

\begin{minipage}[ctb]{\textwidth}
\tt{procedure} \\
	\hs{0.5} [\hyperlink{param}{\tt{param}} \tt{\emph{name [default\_value]}}]* \\
		
	\hs{0.5} \hyperlink{waypoints}{\tt{waypoints}} \\
		\hs{1} [\tt{waypoint \emph{name x y [lat] [lon] [alt]}}]+ \\
		
	\hs{0.5} \hyperlink{blocks}{\tt{blocks}} \\
		\hs{1} [\tt{block \emph{name [description]}}]+ \\
			\hs{1.5} as in \hyperlink{flightplanstructure}{flight plans}... \\
\end{minipage}


%-------------------------------------------------------------------------------
\hypertarget{param}{\subsection{\tt{param}}}
%-------------------------------------------------------------------------------
\begin{quote}
	\tt{param \emph{name [default\_value]}}
\end{quote}

\tt{param} describes the various parameters you can have in the
\hyperlink{procedure}{procedure} wiht their \tt{name} and \tt{default\_value}. For example: \\
\begin{quote}
	\ex{<param name="\qt{rad}" default\_value="\qt{30}"/>}
\end{quote}


%-------------------------------------------------------------------------------
\hypertarget{procedureuse}{\subsection{\tt{procedures} use}}
%-------------------------------------------------------------------------------
\label{procedureuse2}

\tt{procedures} are called with the \hyperlink{include}{\tt{include}} command
in flight plans. There, you specify the \tt{value} of arguments \tt{name}.
You must also specify \hyperlink{deroute}{\tt{deroute}} link with \tt{with}
command where \tt{from} is the name used in the \tt{procedure} and \tt{to}
the \tt{block} to switch to.

\begin{minipage}[ctb]{\textwidth}
Here is an example of the \tt{procedure} inclusion in the flight plan:
\begin{quote}
	\ex{<include name="\qt{hippo1}" procedure="\qt{hippo.xml}" x="\qt{-100}" y="\qt{150}" rotate="\qt{0}">} \\
	\ex{\hs{0.5} <arg name="\qt{alt}" value="\qt{GROUND\_ALT+100}"/>} \\
	\ex{\hs{0.5} <with from="\qt{event1}" to="\qt{another\_block\_of\_flight\_plan}"/>} \\
	\ex{</include>} \\
\end{quote}
\end{minipage}

\begin{minipage}[ctb]{\textwidth}
Here is the corresponding \tt{procedure} \emph{hippo.xml}:
\begin{quote}
	\ex{<procedure>} \\
	\ex{\hs{0.5} <param name="\qt{alt}"/>} \\
	\ex{\hs{0.5} <param name="\qt{radius}" default\_value="\qt{75}"/>} \\
	\ex{\rule{0pt}{1ex}} \\
	\ex{\hs{0.5} <waypoints>} \\
	\ex{\hs{1} <waypoint name="\qt{c1}" x="\qt{0}" y="\qt{0}"/>} \\
	\ex{\hs{1} <waypoint name="\qt{c2}" x="\qt{250}" y="\qt{0}"/>} \\
	\ex{\hs{0.5} </waypoints>} \\
	\ex{\rule{0pt}{1ex}} \\
	\ex{\hs{0.5} <blocks>} \\
	\ex{\hs{1} <block name="\qt{loop}">} \\
	\ex{\hs{1.5} <exception cond="\qt{(RcEvent1())}" deroute="\qt{event1}"/>} \\
	\ex{\hs{1.5} <while cond="\qt{vsupply>100}">} \\
	\ex{\hs{2} <go hmode="\qt{route}" alt="\qt{alt}"} \\ 
	\ex{\hs{2.7} from="\qt{c1}" from\_qdr="\qt{0}" from\_dist="\qt{radius}"} \\
	\ex{\hs{2.7} wp="\qt{c2}" wp\_qdr="\qt{0}" wp\_dist="\qt{radius}"/>} \\
	\ex{\hs{2} <circle wp="\qt{c2}" radius="\qt{radius}" until="\qt{Qdr(180)}" alt="\qt{alt}"/>} \\
	\ex{\hs{2} <go hmode="\qt{route}" alt="\qt{alt}"} \\
	\ex{\hs{2.7} from="\qt{c2}" from\_qdr="\qt{180}" from\_dist="\qt{radius}"} \\
	\ex{\hs{2.7} wp="\qt{c1}" wp\_qdr="\qt{180}" wp\_dist="\qt{radius}"/>} \\
	\ex{\hs{2} <circle wp="\qt{c1}" radius="\qt{radius}" until="\qt{Qdr(0)}" alt="\qt{alt}"/>} \\
	\ex{\hs{1.5} </while>} \\
	\ex{\hs{1.5} <deroute block="\qt{event1}"/>} \\
	\ex{\hs{1} </block>} \\
	\ex{\hs{0.5} </blocks>} \\
	\ex{</procedures>} \\
\end{quote}
\end{minipage}

\begin{minipage}[ctb]{\textwidth}
Here is an example of the call the \tt{procedure} in the flight plan:
\begin{quote}
	\ex{<exception cond="\qt{(RcEvent1())}" deroute="\qt{hippo1.loop}"/>} \\
\end{quote}
\end{minipage}


%===============================================================================
\hypertarget{Ccode}{\section{Remarks about C code}}
%===============================================================================
\label{Ccode2}
Most of the argument are C code. In fact, the xml flight plan file is read and
changed into C code which is compiled and put into autopilot. By this way, you
can write funny flight plan like this:
\begin{quote}
	\ex{<circle wp="\qt{my\_center}" vmode="\qt{climb}" climb="\qt{1}"} \\
	\ex{\hs{1.3} radius="\qt{estimator\_z}" until="\qt{100>vsupply}"/>}
\end{quote}

In this example, the UAV will do circles around \emph{my\_center} by climbing
1 meter per second with a \tt{radius} depending on it's altitude
(\tt{estimator\_z}). This figure is an enlarging helice, which is nice because
you see more and more around your goal \emph{my\_center} and still feel the UAV
is moving by looking at airborne video.
This will end when supply is under 10 volt (\tt{vsupply} is in decivolt). \\

It would also be possible to write
\ex{radius="\qt{2*estimator\_z}"} or
\ex{radius="\qt{50 + 30*sin(stage\_time/28.6)}"} (a three minutes based
sinusoidal radius circle).

Lots of thing are possible by editing airborne C code and play with variables. \\

\emph{Note:} if you want to change the list of user accessible variables or
functions in flight plan, you must change the \tt{functions} and
\tt{variables} lists in \emph{./sw/tools/fp\_syntax.ml} file and
recompile it by tipping \tt{make} in \emph{.sw/tools/}


%===============================================================================
\hypertarget{medit}{\section{\emph{medit} tool}}
%===============================================================================
\label{medit2}

\emph{medit} is a tool to easily defined a flight plan. You can add, move or
clear \tt{waypoints} on a map. You can easily see your various
\hyperlink{blocks}{\tt{blocks}}, their \tt{stages} and their \emph{arguments}.\\
It is really simplier to make or edit a flight plan with \emph{medit} than
with a text editor.


\end{document}
