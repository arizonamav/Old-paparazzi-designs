%
%
%
%   $Id: paparazzi.tex,v 1.1.1.1 2005/01/25 10:57:51 poine Exp $
%   Copyright (C) 2003 Pascal Brisset, Antoine Drouin
%
% This file is part of paparazzi.
%
% paparazzi is free software; you can redistribute it and/or modify
% it under the terms of the GNU General Public License as published by
% the Free Software Foundation; either version 2, or (at your option)
% any later version.
%
% paparazzi is distributed in the hope that it will be useful,
% but WITHOUT ANY WARRANTY; without even the implied warranty of
% MERCHANTABILITY or FITNESS FOR A PARTICULAR PURPOSE.  See the
% GNU General Public License for more details.
%
% You should have received a copy of the GNU General Public License
% along with paparazzi; see the file COPYING.  If not, write to
% the Free Software Foundation, 59 Temple Place - Suite 330,
% Boston, MA 02111-1307, USA.  
% 

%
%
% This may become the paparazzi user manual.
%
%


\documentclass{article}

\usepackage{a4wide}
\usepackage{graphicx}
\usepackage{makeidx}
\usepackage[pagebackref=true,hyperindex=true]{hyperref}

\title{Paparazzi User's Manual}
\author{Pascal Brisset and Antoine Drouin}
\date{\today}

\makeindex

\begin{document}

\maketitle



\begin{abstract}
The system described in this document is an autopilot for model aircrafts.
It consist of custom airborne hardware, a laptop as ground station and a 
retail radio control transmitter for uplink (manual/assisted control, reconfiguration, etc..).
The sensor used are a GPS receiver and infrared thermopiles (melexys mlx90247) 
for horizon sensing. This system is able to fly autonomously a small electro 
powered aircraft. It transmits live video and telemetry data. 
The ground station permits decoding, logging, replay and analysis of these data. 
It also permits airborne code configuration, generation, simulation and flashing 
on target MCU.

\end{abstract}

\section{Description}

\subsection{Architecture : two 8 bits MCUs}

  goals: 
     - maximum availability (modes degrades, manual control)
     - ease of development
  
    logical tasks of increasing conplexity and decreasing importance in separate devices.

  \subsubsection{fly by wire} 
   (avr mega8) responsible for radio control decoding, mixings, servos (c.f. figure ref{fbw}).

  short code, well tested, features similar to a programmable radio control transmitter

  allows manual control and programmed failsafe (radio link loss).


  health monitoring (battery voltage, drawn current etc...)

\
 \subsubsection{autopilot}
 (avr mega128) measures + control loops + telemetry + navigation

  The model can be safely flown in manual mode with only the fly by wire MCU.

  We want to keep and improve this simple system for cheap/small aircrafts.

  We also want to expend it with a third 32 bit processor (like arm or xscale) 
  to get network communications and processing power (FMS like).

\subsection{Technology : cheap and widely available parts}
  4800bps FSK Telemetry signal. Can be fed in most transmitter.
  We use the audio channel of a 50 mW 2.4GHz video transmitter.
  The video Channel is used for real time video. We reach 600m in line of sight.
  The rx antenna is a small patch mounted on top of a helm (cheap self pointing antenna).
  FIXME: ADD PIC

  Modified retail RC receiver : solder a wire after the HF section.

  controller board, sensor, GPS : custom PCBs, all(most) parts smd. 
  home pcbs and home soldering for protos

  Free software for tool (GNU/Linux, gcc, gtk, ocaml....)
  No Windows port known but should be faisible.

\section{Ground station}
  \subsection{hardware}
    Gnu/linux laptop
    
    2.4GHz video receiver + CMX469 modem board  or rtty
    ==FIXME== add pic
    camcorder

  \subsection{obtaining and installing} 

  The source code is available from the Project page ( http://savannah.nongnu.org/projects/paparazzi/ )

  Use the anonymous CVS server to get the up to date source code and documentation:

{\em  export CVS\_RSH="ssh"}
{\em  cvs -z3 -d:ext:anoncvs@savannah.nongnu.org:/cvsroot/paparazzi co paparazzi2}

  You can also download a tarball from this website ( http://www.recherche.enac.fr/paparazzi/paparazzi.tar.gz ). 
    Debian sarge users can get the required extra packages from there ( http://www.recherche.enac.fr/paparazzi ). 

  ***********************************************************************
  Set the PAPARAZZI\_HOME environment variable to the top directory
  of the distribution (this variable is used by some of the components).
  ***********************************************************************
  {\em export PAPARAZZI\_HOME=/some/dir/paparazzi2 }
Default configurations files (in conf/ directory) should allow to
compile both embedded and ground software:
\begin{itemize}
 \item 1) HAVE A LOOK at conf/Makefile.local

 \item  2) Create conf/conf.xml and the related files for your convenience
("make configure" runs a graphics interface which may help; however
this gui is in a very early alpha stage). Some examples are provided.

  \item 3) "make" in top directory should compile everything

\end{itemize}

\subsection{The ivy software bus}



\subsection{configuration interface}
  This windowed program allows to graphicaly edit the configuration of a Paparazzi.
  It also can also be used to program your controller board with various test and calibration
  programs.
  type {\em{make configure}} in the source top directory.

  \subsection{telemetry interface : recording and display}	
  \index{telemetry}
	receive : retrieve telemetry data, store them on disk and broadcast them over a network
        gui : display telemetry data. Can be feed live by the receive programm or by the replay programm

	map calibration :
	 uses 3 points. trivial projection but sufficient for short range.
  \subsection{replay interface}

  \subsection{hitl simulator}
    Airborn programs runs on their target MCUs. 
    Their inputs and outputs (GPS, infrared and servos) are bypassed to the laptop.
    A dumb flight model allows debugging and non regression testing. Also useful for 
    tunnig navigation
 
\section{Airborne software}
  \subsection{Fly by wire}



    only supports PPM - subject to jamming - filtering - would be better with a PCM encoding

\begin{figure}
\includegraphics[width=15cm]{fly_by_wire}
\caption{\label{fbw}Fly by wire data processing}
\end{figure}


 \subsection{Autopilot}

   \subsubsection{Low level control loop}
     20Hz . P controller for pitch and roll 
     PI for throttle
     
     mostly unfiltered attitude data from infrared sensor + GPS climb rate for throttle


   \subsubsection{Navigtion loop}
     1Hz P controllers on heading to waypoint and altitude
     GPS data


   \subsubsection{Infrared calibration}
      contrast
      LLS
      

   \subsubsection{FMS}
      modes (auto1 auto2)
      waypoints circling
      waypoint crossing
      mode home
      automatic take off

\section{Assembling boards}
PCBS : homebuild eurocircuits

see part list

solder one(a group of) component at a time. test with voltmeter or scope
use provided programms. 

mcu fuses - used to define type of clock - factory supplied with 1MHz internal oscillator -
must switch to {\em{ceramic resonator}} for ``fly by wire''. He will be generating clock for autopilot.
``autopilot'' will have to be programmed to ``external clok''. If you mess up, you can make a zombie 
out of your MCU unless you can provide the awaited signal or crystal. You can read current fuses configuration by typing {\em{make read\_fuses}} in an avr source directory. The correct values are contained
in the Makefile and can be programmed by typing {\em{make wr\_fuses}}. 
The graphical configurator also allows these operations.


mcu flashing - The method (and corresponding wiring) we use for flashing and fuses programming is 
called serial (SPI) programming. It is possible to programm a resident bootloader who will take care
of following programmations using serial RS232

\subsection{power supply}

\subsection{pc link}
  This board is a level converter. It converts between the TTL 5V of the controller board and respectively, the 
  3.3V of the parallel port and the 10V of the rs232 port.
  The parallel port is used for SPI programming of MCUs. The rs232 port are used fo serial 
  communications with MCUs, for example during simulations.
  This board is meant to stay on ground.
 

\subsection{controller board}
  solder the fly by wire MCU (mega8), its crystal and the programmation socket
  Connect to a current limited power supply and check current.

  build a wire harness to the pc link board
  plug the pc link in your parallel port
  try connecting to MCU in serial programming mode (SPI) (button on gui)  
    FIXME: if it fails
  programm the fuses of the MCU (describe the crystal connected to the mcu) (button on gui)
    check crystal oscillating with scope if available.

  try programming the uart test
  plug a straight serial cable in the serial1 connector of the pc link and the other end in one of your computer rs232 port. If your computer doesn't have any, use a usb to rsr232 converter. 
 you should see a message comming from your board telling you the link is ok. Check the other direction, writing to the board.

 solder servo driver and connector (maybe later...)
   run the test programm (servo calibration)

 find ppm signal and supply in receiver. solder wire (computer cdrom wire)
 solder the other end to the controller board.
  run the test programm (radio calibration)
  

 solder the autopilot MCU
  try uart 1
  
  try spi (write a test with SPI and UART)

  solder the modem
   try modem (in line input - with rtty ?? )



\subsection{ground modem}
  same story with mega8 and crystal
  check connection, write fuse, program serial test
  solder modem
  connect to airborne modem
  watch telemetry

\subsection{infrared sensor}
  solder amp, resistor and capa and thermopiles.
  connect to autopilot MCU ADCs 
  watch telemetrie values;

\section{Fitting system in the airframe}

All the processing available on programmable radio transmitters (travel adj, mixing etc..) are here 
done by the fly by wire MCU. This is cool because you don't need to change your transmitter programm 
when you change aircraft, but it also enables the autopilot to use these features.

\subsection{radio control transmitter calibration}
tab in gui.
use the default programm of your rc transmitter with travels set to 100\% and trim centered.
programm the controller board (actually fly by wire MCU) with the test programm (button on gui).
if everything goes well you will see values of the channels in the signal send by your transmitter.
record min max neutral for each channel and setup control
give it a name  
generate a configuration file.


\subsection{servos travel and mixer setup}

Mount the board in the airframe and connect servos. programm the board with the servo setting programm (tab in gui)
for each servo, define name of the control, travel neutral and direction. 
Try to use maximal travel and long control arms.

\subsection{infrared sensor}
describe the way to mout it on the airframe and the needed configuration.


\section{Simulation}
Don't attempt to fly your aircraft until you've succesfully simulated with your configuration and learned how the system works. There are two type of simulation :

\subsection{``Software in loop'' simulation}
needs no hardware but the laptop. 
It is great to learn while you are building the hardware.
Thanx to our magnificient C compiler, we are able to compile the same code for the AVR mcu and for the i386 laptop.

\subsection{``Hardware in the loop'' simulation}



\section{Test Flight}

\subsection{checklist}

This checklist applies to our twinstar.

Switch ground station on - connect modem - launch receive and gui - check modem messages + check ground batterie

Switch rc transmitter on - check programm - all switches pushed - mode auto1 - throttle low

Switch airplane on - check model name and rc transmitter name

check "waiting calibration" on ground station - switch to mode manual

Put airplane on nose - push roll stick - check contrast on ground station

switch briefly to full throttle to trigger speed controllers

Check command direction and travel

Switch to auto1. Check corrections direction (if you put the plane nose down, the elevator should raise, if you bank to the right, the left aileron should raise).

Check GPS status on ground station.

Flight briefing - check mission on map

check autopilot mode

take off. For automatic take off (auto 2), full throttle will signal take off and trigger full throttle.


\subsection{adjusting trim}


This first flight is flown in manual mode. It is used to trim the airframe and get an estimation of infrared neutrals.
It is very important that you trim your model perfectly. Choose a day without wind or turbulence. Fly long 
straight lines trying not to touch your sticks.

Watch your batterie voltage on the laptop (ask someone or use a vocal synthetiser).

After the flight
      offset servos to recenter your rc transmitter trims.
      get an estimation of contrast\_gain (ir\_gain = contrast\_gain/contrast) comparing your contrast measure and lls
      get an estimation of infrared neutrals (pitch and roll) (play the telemetry record during straight lines or maybe       write a tool).
      get an estimation of throttle for level flight
      watch the record for anomalies (describe)
      plot parameters (like airspeed, climb rate, current consumption...). This is a great tool for tunning an airframe.



\subsection{adjusting low level loop (attitude loop)}
In this flight you will adjust the infrared neutrals and low level loop gains.
The number of parameter that you are able to tune in a single flight depends on how many switches and sliders are available on your rc transmitter. With the Multiplex MC3030 (9 channels) we have two spare sliders and one three positions 
switch. It allows us to tune 4 parameters at a time.  

Update you airframe description. Use the values from the previous flight for infrared neutral and contrast\_gain. Use low values for the low level loop P gains. Reflash your airplane.

For this flight we will programm an autopilot mode in auto1 which will hold the plane in an attitude described by the 
roll and pitch sticks. If you leave your sticks centered, the plane will fly level. If you push your roll stick, the plane will bank to a given value (full travel -> 30�) and stay in this attitude.


Take off in manual mode. Gain altitude .
Check that the plane is flying level and that your tranmistter trims are centered.
If this is not the case, redo the programm of the previous flight.

Engage auto1 mode. Be ready to switch back to manual if the plane doesn't react like you expect.
Switch to neutral calibration and fine tune values so that the plane flies level. 
This stage is very important for the navigation to work.
Raise the values of the low level loop P gains until the plane reacts quickly to an attitude change but 
without oscillating. 

Now your plane should be capable of holding attitude. This is a very strange feeling for the pilote.


After the flight: 
      plot params
      fine tune your contrast gain with LLS measure
      update low level P gains and infrared neutrals in model description.


\subsection{adjusting autopilot gains}

In this flight, we will programm a navigation mode in the auto2 bank which will navigate the airplane around waypoints.

describe a mission: home and waypoints 
 You can click on a map (needs calibration) or walk with the plane GPS to find the position of your points.
 
 For our twinstar, the typical missions are two waypoints distant from 300m and at 80m above ground level.

 Flash your airplane. If you have a map, check that the mission that the airplane transmits on boot shows up on the map at the right place.

 After the checklist, check that the position transmitted is near your home (relative position - NAV message) and that the coordinates go in the right direction (X->NORTH Y->EAST FIXME:check!!)

Take of in manual mode - check plane level and rc trim centered
Switch to auto1 - check plane level
Switch to auto2 - adjust nav P gain and max bank angle for smooth nav



Make more flights taking off in manual - 
When your are confident you can take off in auto1 or auto2. Specify desired climb rate and security altitude



\section{Part list and supplyers}

\subsection{power supply}

\subsection{controller board}


\subsection{infrared sensor}
\begin{tabular}{| l | l | l | l |}
\hline
4 & thermopiles MLX90247ESF-B & www.digikey.com (PN: MLX90247ESF-B-ND) & 16,5 euros each, 12.5 euros >= 10 \\
\hline
1 & op amp AD8552 TSSOP case & le fabriquant envoie des �chantillons - formulaire sur leur site.&  \\
\hline
1 & MOLEX 6 contacts connector 1.25mm pitch & radiospares (PN: 53047-0610) & by 10 - 2,34 euros \\
\hline
\end{tabular}


r�sistances et condensateurs CMS





\section{glossaire}
\begin{description}
  \item[ADC] Analog to Digital Converter: A chip or MCU peripheral that converts an analog voltage to its binary representation.

  \item[CPU] Control Processor Unit:


  \item[MCU] Micro Controller Unit: A chip containing a CPU and varous peripherals like memory, io ports, timer, ADCs etc..
                             The paparazzi controller board uses two of these chip.  


\item[LLS] Linear Least Square: 

\end{description}

\printindex


\end{document}
